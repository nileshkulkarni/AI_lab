\documentclass{beamer}
%
% Choose how your presentation looks.
%
% For more themes, color themes and font themes, see:
% http://deic.uab.es/~iblanes/beamer_gallery/index_by_theme.html
%
\mode<presentation>
{
  \usetheme{default}      % or try Darmstadt, Madrid, Warsaw, ...
  \usecolortheme{default} % or try albatross, beaver, crane, ...
  \usefonttheme{default}  % or try serif, structurebold, ...
  \setbeamertemplate{navigation symbols}{}
  \setbeamertemplate{caption}[numbered]
} 

\usepackage[english]{babel}
\usepackage[utf8x]{inputenc}

\title[PTA]{Perceptron Training Algorithm}
\author{Sushant | Nilesh | Vipul | Anand}
\institute{Indian Institute of Technology, Bombay}


\begin{document}

\begin{frame}
  \titlepage
\end{frame}


% Uncomment these lines for an automatically generated outline.
\begin{frame}{Outline}
  \tableofcontents
\end{frame}
\begin{frame}{PTA Convergence}
\section{PTA Convergence}
	PTA converges for these linearly seperable Operators 
    \begin{itemize}
    	\item Nand : Weights are [-2, -3, -4]
        \item Nor : Weights are [-2, -2, -1]
        \item 5 bit Majority : Weights are [3, 3, 4, 4, 4, 9]
     \end{itemize}     

\end{frame}

\begin{frame}{Cycle Detection and Non-convergence}
\section{Cycle Detction}
From theory we know that power of Peceptron is limited by the linear seprability of the data set. If the data set is linearly seprable by a hyperplane in the dimesion of the data, then we can find a trained perceptron for it. And the PTA will converge.

So we have some examples in which Perceptrons cannot be used are:

\begin{itemize}
	\item 5 bit Palindrome : Initial Iterations: 20 repitition rate 4 
    \item 5 bit Parity : Initial Iterations: 23 repitition rate 4 
    \item XOR : Initial Iterations: 9 repitition rate 4 
\end{itemize}
\end{frame}
\end{document}
