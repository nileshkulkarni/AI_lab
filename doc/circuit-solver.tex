\documentclass{beamer}
%
% Choose how your presentation looks.
%
% For more themes, color themes and font themes, see:
% http://deic.uab.es/~iblanes/beamer_gallery/index_by_theme.html
%
\mode<presentation>
{
  \usetheme{default}      % or try Darmstadt, Madrid, Warsaw, ...
  \usecolortheme{default} % or try albatross, beaver, crane, ...
  \usefonttheme{default}  % or try serif, structurebold, ...
  \setbeamertemplate{navigation symbols}{}
  \setbeamertemplate{caption}[numbered]
} 

\usepackage[english]{babel}
\usepackage[utf8x]{inputenc}
\usepackage{tikz}
\usepackage{graphicx}



\title[Circuit Solver]{Circuit Solver}
\author{Anand, Sushant, Nilesh , Vipul}
\institute{IIT Bombay}

\begin{document}

\begin{frame}
  \titlepage
\end{frame}

\begin{frame}{Table of Contents}
  \tableofcontents
\end{frame}

\section{Introduction}

\begin{frame}{Introduction}
The curcuit involves solving circuits using prolog.
\vskip 1cm
Prolog has its roots in first-order logic, a formal logic, and unlike many other programming languages, Prolog is declarative: the program logic is expressed in terms of relations, represented as facts and rules.

\end{frame}

\section{Circuit Simulation}

\subsection{Basic Gates}

\begin{frame}{Basic Gates}
To implement the basic gates using prolog, we used the following strategy:
\newline
\begin{itemize}
\item For a particular gate, we enumerated the cases when it goes to \texttt{true} and when it goes to {false}.
\item We made sure to use combination of complements to enumerate all possible cases for a particular gate
\item Finally, we did a union of all required gates to get the final answer.
\end{itemize}
More information of this in the coming slides.
\end{frame}

\subsection{API}

\begin{frame}{API}
Taking help from the slides on course webpage, we formulated the following API:
\begin{itemize}
\item \texttt{signal(t,X): t}  here is a terminal which takes the values of \texttt{true} or \texttt{false} which are stored in \texttt{X}.

\item \texttt{type(G,x): x} here is the type of \texttt{gate G} and takes the values of AND, OR, NOT etc

\item  \texttt{in(S,X,G): X} is the input terminal corresponding to \texttt{gate G} at \texttt{serial number S}
\item \texttt{out(T,G) T} is the output terminal for \texttt{gate G}.
\end{itemize}
This basic API has been implemented in \texttt{circuit.pl}. This is useful
\end{frame}


\section{Applications}
\subsection{Majority}
\begin{frame}{Majority}
\textbf{Problem Statement:} A majority gate returns true if and only if more than 50\% of its inputs are true. We have improved it to the following definition:
\newline
\texttt{majority(m,n)} This is true if number of 1's is greater than \texttt{n} in a \texttt{m} bit number.
\newline \newline
\textbf{Usage}: 
To simplify things and to avoid manually feeding of large inputs, we wrote 2 \texttt{c++ programs}: \newline \texttt{majority} takes input as \texttt{5  3} and generates a expression of unions and intersections.
This expression is then passed to \texttt{init}, this program generates set of \texttt{rules} which will be given to the prolog program.
These set of rules alongwith the basic prolog program, help us solve the majority circuit.
\end{frame}

\begin{frame}{Code Structure}
\begin{itemize}
\item\texttt{majority.pl}
\newline
The code is built upon the basic API mentioned during the start.
Apart from that, a new rule is added.
\newline
\texttt{valid(V1,V2,V3,V4,V5)} makes sure that all the variables take values either 0 or 1 and are not undefined in any state.
\item \texttt{majority} \newline This binary essentially converts the input into an expression consisting of unions and intersections.
\newline It uses a recursive relationship to generate the expression:
\texttt{majority(m,k) = majority(m-1,k) + majority(m-1,k-1)}
\item \texttt{init}  \newline
This just essentially creates the entire circuit using the API by the passing the expression generated in \texttt{majority}.
\end{itemize}

\end{frame}


\begin{frame}{Use Cases}
Following are sample use cases which can be used to test the majority circuit.
\newline
\begin{itemize}
\item \texttt{signal(1,1,1,0,0,t17,X)} This returns the value of terminal \texttt{17} in \texttt{X}
\item \texttt{signal(X1,X2,X3,X4,X5,t17,1)}: This is keeps on giving values of inputs X1-5 when the terminal 17 has value 1.
\end{itemize}
\end{frame}


\subsection{Palindrome}
\begin{frame}{Palindrome}
\textbf{Problem Statement} Should return \texttt{true} if the input follows the property of palindromes.
\newline \newline
\textbf{Usage}: 
To simplify things and to avoid manually feeding of large inputs, we wrote a \texttt{c++ program}: \texttt{init}, this program generates set of \texttt{rules} which will be given to the prolog program according to an expression, which is of the form 
These set of rules alongwith the basic prolog program, help us solve the palindrome circuit.
\end{frame}

\begin{frame}{Code Structure}
\begin{itemize}
\item\texttt{palindrome.pl}
\newline
The code is built upon the basic API mentioned during the start.
Apart from that, a new rule is added.
\newline
\texttt{valid(V1,V2,V3,V4,V5)} makes sure that all the variables take values either 0 or 1 and are not undefined in any state.
\item \texttt{init}  \newline
This just essentially creates the entire circuit using the API by the passing the expression generated in \texttt{majority}.
\end{itemize}

\end{frame}



\begin{frame}{Use Cases}
Following are sample use cases which can be used to test the palindrome circuit.
\newline
\begin{itemize}
\item \texttt{signal(1,1,1,0,0,t3,X)} This returns the value of terminal \texttt{3} in \texttt{X}
\item \texttt{signal(X1,X2,X3,X4,X5,t3,1)}: This is keeps on giving values of inputs X1-5 when the terminal 3 has value 1.
\end{itemize}
\end{frame}



\end{document}
