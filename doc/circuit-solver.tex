\documentclass{beamer}
%
% Choose how your presentation looks.
%
% For more themes, color themes and font themes, see:
% http://deic.uab.es/~iblanes/beamer_gallery/index_by_theme.html
%
\mode<presentation>
{
  \usetheme{Darmstadt}      % or try Darmstadt, Madrid, Warsaw, ...
  \usecolortheme{default} % or try albatross, beaver, crane, ...
  \usefonttheme{default}  % or try serif, structurebold, ...
  \setbeamertemplate{navigation symbols}{}
  \setbeamertemplate{caption}[numbered]
} 

\usepackage[english]{babel}
\usepackage[utf8x]{inputenc}
\usepackage{tikz}
\usepackage{graphicx}



\title[Circuit Solver]{Circuit Solver}
\author{Anand, Sushant, Nilesh , Vipul}
\institute{IIT Bombay}

\begin{document}

\begin{frame}
  \titlepage
\end{frame}

\section{Introduction}

\begin{frame}{Introduction}
The curcuit involves solving circuits using prolog.
\vskip 1cm
Prolog has its roots in first-order logic, a formal logic, and unlike many other programming languages, Prolog is declarative: the program logic is expressed in terms of relations, represented as facts and rules.

\end{frame}

\section{Circuit Simulation}

\subsection{Basic Gates}

\begin{frame}{Basic Gates}
To implement the basic gates using prolog, we used the following strategy:
\newline
\begin{itemize}
\item For a particular gate, we enumerated the cases when it goes to \texttt{true} and when it goes to {false}.
\item We made sure to use combination of complements to enumerate all possible cases for a particular gate
\item Finally, we did a union of all required gates to get the final answer.
\end{itemize}
More information of this in the coming slides.
\end{frame}

\subsection{API}

\begin{frame}{API}
Taking help from the slides on course webpage, we formulated the following API:
\begin{itemize}
\item \texttt{signal(t,X): t}  here is a terminal which takes the values of \texttt{true} or \texttt{false} which are stored in \texttt{X}.

\item \texttt{type(G,x): x} here is the type of \texttt{gate G} and takes the values of AND, OR, NOT etc

\item  \texttt{in(S,X,G): X} is the input terminal corresponding to \texttt{gate G} at \texttt{serial number S}
\item \texttt{out(T,G) T} is the output terminal for \texttt{gate G}.
\end{itemize}
This basic API has been implemented in \texttt{circuit.pl}. This is useful
\end{frame}


\section{Applications}
\subsection{Majority}
\begin{frame}{Majority}
\textbf{Problem Statement:} A majority gate returns true if and only if more than 50\% of its inputs are true. We have improved it to the following definition:
\newline
\texttt{majority(m,n)} This is true if number of 1's is greater than \texttt{n} in a \texttt{m} bit number.

\textbf{Use Cases}: 
\end{frame}
\end{document}
