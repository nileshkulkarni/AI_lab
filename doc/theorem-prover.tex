\documentclass{beamer}
%
% Choose how your presentation looks.
%
% For more themes, color themes and font themes, see:
% http://deic.uab.es/~iblanes/beamer_gallery/index_by_theme.html
%
\mode<presentation>
{
  \usetheme{Darmstadt}      % or try Darmstadt, Madrid, Warsaw, ...
  \usecolortheme{default} % or try albatross, beaver, crane, ...
  \usefonttheme{default}  % or try serif, structurebold, ...
  \setbeamertemplate{navigation symbols}{}
  \setbeamertemplate{caption}[numbered]
} 

\usepackage[english]{babel}
\usepackage[utf8x]{inputenc}
\usepackage{tikz}
\usepackage{graphicx}



\title[Theorem Prover]{Theorem Prover}
\author{Anand, Sushant, Nilesh , Vipul}
\institute{INDIA}

\begin{document}

\begin{frame}
  \titlepage
\end{frame}

% Uncomment these lines for an automatically generated outline.
\begin{frame}{Outline}
  \tableofcontents
\end{frame}

\section{Heuristic}

\begin{frame}{Heuristic}

We assume that the target formula(that needs to be proved) is F, after simplification using Deduction Theorem.

Suppose the Hypothesis set contains a formula of the sort $(A_{1}\rightarrow(A_{2}\rightarrow(A_{3}.......(A_{n}\rightarrow F)..)$ 
\newline
Denote rest of the Hypothesis vector by H.
Divide the problem into subproblems of the following kind.
\begin{itemize}
  \item H $\vdash$ $A_{1}$
  \item H, $A_{1}$ $\vdash$ $A_{2}$
  \item In general H , $A_{1}$, $A_{2}$ .... $A_{k}$ $\vdash$ $A_{k+1}$
\end{itemize} 

\end{frame}


\begin{frame}{Heuristic}


Before we proceed, we note the following. If there is a proof for \newline A $\vdash$ B, there is infact a proof from first principles for B starting with A as the Hypothesis. The exact proof sequence might not be explicit. All the theorem says is that there exist such a sequence.
\newline

If we are able to solve each of these problems seperately, then we have infact solved the original problem in the following way:


\end{frame}

\begin{frame}{Heuristic}




\begin{itemize}
  \item From Deduction theorem, since we have shown that H $\vdash$ $A_{1}$, we know that there is a proof for $A_{1}$ from H using first principles. 
  \item We also showed that  H, $A_{1}$ $\vdash$ $A_{2}$. To the proof above, we append the proof trace for this problem.
  \item In general, we have showed that $A_{1}$, $A_{2}$ .... $A_{k}$ $\vdash$ $A_{k+1}$Thus, we append the proof  trace for this to the earlier proofs. 
  \item Finally, Using the above steps we would have added \newline $A_{1}$, $A_{2}$ .... $A_{n}$ to our Hypothesis set. Through a series of MP inferences, the original problem can be easily proved.
\end{itemize} 


\vskip 1cm



\end{frame}

\begin{frame}{Heuristic: Example}


\subsection{Example}
Consider the following problem. \newline

\begin{center}
$\vdash$(((P$\rightarrow$(Q$\rightarrow$F))$\rightarrow$F)$\rightarrow$((P$\rightarrow$F)$\rightarrow$Q)), 
\end{center}


which is the expanded form of  
\begin{center}
 $\vdash$ ((P $\wedge $Q)$\rightarrow$(P$\vee$Q)) 
\end{center}



Simplifying the problem using deduction theorem we obtain,
\begin{center}
((P$\rightarrow$(Q$\rightarrow$F))$\rightarrow$F), 
(P$\rightarrow$F), 
(Q$\rightarrow$F) $\vdash$ F. \newline
\end{center}


The first formula in the Hypothesis set is ((P$\rightarrow$(Q$\rightarrow$F))$\rightarrow$F), which is of the form $(A_{1}\rightarrow(A_{2}\rightarrow(A_{3}.......(A_{n}\rightarrow F)..)$ 


\end{frame}
\begin{frame}{Heuristic}


Applying our Heuristics we obtain, the following subproblems(s):
  \begin{itemize}
    \item (P$\rightarrow$F),  (Q$\rightarrow$F) $\vdash$ (P$\rightarrow$(Q$\rightarrow$F))
  \end{itemize}


In this case the formula is of type  ($A_{1}$ $\rightarrow$ F), so we have only one subproblem.
Let's focus on this subproblem for the moment. Simplifying the subproblem using deduction theorem we obtain,

\begin{center}
(P$\rightarrow$F),  (Q$\rightarrow$F), P , Q $\vdash$ F \newline
\end{center}

This can be easily proved by a Modus ponus inference, thereby proving our original problem.
\end{frame}




\begin{frame}{Heuristic: Another Example}


\subsection{Example}
Consider another problem. \newline

\begin{center}
$\vdash$(((P$\rightarrow$(((Q$\rightarrow$F)$\rightarrow$R)$\rightarrow$F))$\rightarrow$F)
$\rightarrow$
\newline
((((P$\rightarrow$(Q$\rightarrow$F))$\rightarrow$F)$\rightarrow$F)$\rightarrow$((P$\rightarrow$(R$\rightarrow$F))$\rightarrow$F)))
, 
\end{center}


which is the expanded form of  
\begin{center}
 $\vdash$ ((P$\wedge$(Q$\vee$R))$\rightarrow$((P$\wedge$Q)$\vee$(P$\wedge$R)))
\end{center}



Simplifying the problem using deduction theorem we obtain,
\begin{center}
(P$\rightarrow$(R$\rightarrow$F)), \newline
((P$\rightarrow$(((Q$\rightarrow$F)$\rightarrow$R)$\rightarrow$F))$\rightarrow$F), \newline
(((P$\rightarrow$(Q$\rightarrow$F))$\rightarrow$F)$\rightarrow$F) $\vdash$ F. \newline
\end{center}


The third formula in the Hypothesis set is (((P$\rightarrow$(Q$\rightarrow$F))$\rightarrow$F)$\rightarrow$F), which is of the form $(A_{1}\rightarrow(A_{2}\rightarrow(A_{3}.......(A_{n}\rightarrow F)..)$ 


\end{frame}
\begin{frame}{Heuristic: Another Example}


Applying our Heuristics we obtain, the following subproblems(s):
  \begin{itemize}

\item 
\begin{center}

(((P$\rightarrow$(Q$\rightarrow$F))$\rightarrow$F)$\rightarrow$F),
(P$\rightarrow$(R$\rightarrow$F)) $\vdash$ (P$\rightarrow$(((Q$\rightarrow$F)$\rightarrow$R)$\rightarrow$F))

\end{center}

\item 
\begin{center}
(P$\rightarrow$(R$\rightarrow$F)) ,
((P$\rightarrow$(((Q$\rightarrow$F)$\rightarrow$R)$\rightarrow$F))$\rightarrow$F)
$\vdash$
(((P$\rightarrow$(Q$\rightarrow$F))$\rightarrow$F)$\rightarrow$F)
\end{center}

  \end{itemize}

In this case, the formulas are of type  ($A_{1}$ $\rightarrow$ F), so we have only one subproblem for each formula.


\end{frame}








\begin{frame}{Heuristic: Another Example}
  Simplifying the problems using deduction theorem, we obtain
 
\begin{itemize}

\item 
((Q$\rightarrow$F)$\rightarrow$R),
(P$\rightarrow$(R$\rightarrow$F)),
P,
(((P$\rightarrow$(Q$\rightarrow$F))$\rightarrow$F)$\rightarrow$F)
 $\vdash$ 
 F

\item 
(P$\rightarrow$(Q$\rightarrow$F)),
(P$\rightarrow$(R$\rightarrow$F)),\newline
\hspace{10 mm}
((P$\rightarrow$(((Q$\rightarrow$F)$\rightarrow$R)$\rightarrow$F))$\rightarrow$F)
$\vdash$
F
  \end{itemize}
\end{frame}


\begin{frame}{Heuristic: Another Example}

Let's focus on the first subproblem for the moment. 
The third formula in it's Hypothesis set  (((P$\rightarrow$(Q$\rightarrow$F))$\rightarrow$F)$\rightarrow$F) is of the form
($A_{1}$ $\wedge$ F).

This can be easily proved by a Modus ponus inference, thereby proving our original problem.
\end{frame}












\section{And-Or Trees}
\begin{frame}{And-Or Trees}
 And-or trees are helpful for problem-reductions. Each node in the tree is a seperate problem.
 We define an "AND"($\wedge$) node and a "OR"($\vee$) node. \newline
   \subsection*{$\wedge$ nodes}
   A node is an \textbf{$\wedge$ node} if solving all of it's children nodes solves the problem specified by the node.
   \newline
   \subsection*{$\vee$ nodes}
  A node is a \textbf{$\vee$ node} if solving any of it's children nodes solves the problem specified by the node. 
  \newline
  \subsection*{Elementary nodes}
  A node is an \textbf{Elementary node} if it can be solved without any further problem reductions.
\end{frame}

  


\begin{frame}{And-Or Trees: Example}
\begin{tikzpicture}
        
        \node[circle,draw] (p) at (11,6) {$\wedge$};
        
        
        \node[circle,draw] (p1) at (8,4) {$\wedge$};
        \node[circle,draw] (p2) at (6,2) {$n_{1}$};
        \node[circle,draw] (p3) at (10,2) {$n_{2}$};
        
        
        \node[circle,draw] (p4) at (14,4) {$\vee$};
        \node[circle,draw] (p5) at (12,2) {$n_{3}$};
        \node[circle,draw] (p6) at (16,2) {$n_{4}$};
        
		\path 
        (p) edge              node {} (p1)            
        (p) edge              node {} (p4)
        
        (p1) edge              node {} (p2)            
        (p1) edge              node {} (p3)
        
        (p4) edge              node {} (p5)             
        (p4) edge              node {} (p6);
 \end{tikzpicture}
	\vskip 1cm
   
 The nodes in the tree correspond to different problems. The root is our original problem. $n_{1}$ , $n_{2}$, $n_{3}$ and $n_{4}$ in the tree correspond to elementary nodes.\newline
  
\end{frame}

  


\section{Building And-Or Trees}

\begin{frame}{Building And-Or Trees}

\begin{itemize}
\item
 The first step of our algorithm is to build an And-or tree. To achieve this, we create a root which is our original problem. The root is initially an OR node. 
\item 
 For every formula of the form $(A_{1}\rightarrow(A_{2}\rightarrow(A_{3}.......(A_{n}\rightarrow F)..)$ in the Hypothesis set of the root, we create a new child for the root. This child is an AND node. 
\item 
 For each subproblems for this child, we create an OR node for each suproblem pertaining to this formula.
\item
 For every unexpanded OR node in the tree, recursively do the above steps.
 
\end{itemize}	

\end{frame}



\begin{frame}{Evaluating the Tree}
  To evaluate the tree, the elementary problems are run for a few steps, and then the result is propagated upwards. If the root is unsolved till now, we prompt the user for human help. 

\end{frame}

\end{document}
