\documentclass{beamer}
%
% Choose how your presentation looks.
%
% For more themes, color themes and font themes, see:
% http://deic.uab.es/~iblanes/beamer_gallery/index_by_theme.html
%
\mode<presentation>
{
  \usetheme{default}      % or try Darmstadt, Madrid, Warsaw, ...
  \usecolortheme{default} % or try albatross, beaver, crane, ...
  \usefonttheme{default}  % or try serif, structurebold, ...
  \setbeamertemplate{navigation symbols}{}
  \setbeamertemplate{caption}[numbered]
} 

\usepackage[english]{babel}
\usepackage[utf8x]{inputenc}

\title[Your Short Title]{A* Search Algorithm: An Analysis}
\author{Sushant | Nilesh | Vipul | Anand}
\institute{Indian Institute of Technology, Bombay}


\begin{document}

\begin{frame}
  \titlepage
\end{frame}

% Uncomment these lines for an automatically generated outline.
\begin{frame}{Outline}
  \tableofcontents
\end{frame}

\section{Introduction}

\subsection{Code Structure}
\begin{frame}{Code Structure}

Following classes were made to make the A* modular:
\begin{itemize}
  \item Node
  \item Graph
  \item AStar
  \item Graph
\end{itemize}
Apart from these, corresponding to each example a class was made:
\begin{itemize}
  \item 8puzzle
  \item Missionary
  \item bidirectional

\end{itemize}
\end{frame}

\subsection{Data Structures}

\begin{frame}{Data Structures}

The following datastructures were implemented:
\begin{itemize}
\item Used multiset to implement open set and closed set
\item Hash Map
\end{itemize}

We've implemented parent-pointer redirection as well.
It is essentially used when a non monotonic heuristic is used.
The pseudo codes is present on the next couple of slides.
% Commands to include a figure:
%\begin{figure}
%\includegraphics[width=\textwidth]{your-figure's-file-name}
%\caption{\label{fig:your-figure}Caption goes here.}
%\end{figure}
\end{frame}

\subsection{Psuedo Code}
\begin{frame}[fragile]{Psuedo Code}
\begin{verbatim}
function A*(start,goal)
    closedset := the empty set   
    openset := {start}   
    came_from := the empty map 
 
    g_score[start] := 0 
    f_score[start] := g_score[start] + 
    heuristic_cost_estimate(start, goal)
 
    while openset is not empty
        current := the node in openset 
            having the lowest f_score[] value
        if current = goal
            return reconstruct_path(came_from, goal)
 
        remove current from openset
        add current to closedset
\end{verbatim}
\end{frame}

\begin{frame}[fragile]
\begin{verbatim}
        for each neighbor in neighbor_nodes(current)
            if neighbor in closedset
                continue
            tentative_g_score := g_score[current] + 
            dist_between(current,neighbor)
 
            if neighbor not in openset or 
              tentative_g_score < g_score[neighbor] 
                came_from[neighbor] := current
                g_score[neighbor] := tentative_g_score
                f_score[neighbor] := g_score[neighbor] + 
                heuristic_cost_estimate(neighbor, goal)
                
                if neighbor not in openset
                    add neighbor to openset
 
    return failure
\end{verbatim}
\end{frame}
\begin{frame}{Readable Mathematics}

Let $X_1, X_2, \ldots, X_n$ be a sequence of independent and identically distributed random variables with $\text{E}[X_i] = \mu$ and $\text{Var}[X_i] = \sigma^2 < \infty$, and let
$$S_n = \frac{X_1 + X_2 + \cdots + X_n}{n}
      = \frac{1}{n}\sum_{i}^{n} X_i$$
denote their mean. Then as $n$ approaches infinity, the random variables $\sqrt{n}(S_n - \mu)$ converge in distribution to a normal $\mathcal{N}(0, \sigma^2)$.

\end{frame}

\end{document}
